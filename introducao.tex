\chapter{Introdução}

Este capítulo contextualiza o tema deste trabalho, apresentando motivações para o uso da programação orientada a aspectos e a necessidade de realizar
a modelagem de sistemas orientados a aspectos. Apresentam-se os objetivos, a hipótese, a metodologia da pesquisa e a justificativa para realização do
estudo. Ao final, apresentam-se os resultados obtidos, as limitações e a organização desta dissertação.

\section{Contextualização}

A Programação Orientada a Objetos (POO) \sigla{POO}{Programação Orientada a Objetos} é um paradigma de programação amplamente difundido e utilizado
para o desenvolvimento de aplicações. Esse paradigma permite implementações em um nível mais alto de abstração e o reuso de comportamentos
\cite{Laddad:2003:AAP:993468}.

O desenvolvimento de um sistema utilizando POO geralmente é dividido em quatro fases: análise, projeto, implementação e testes
\cite{pressman:01}. Durante as fases de análise e projeto eliciam-se os requisitos e elabora-se a modelagem do sistema. Esta modelagem deve expressar 
características estruturais e dinâmicas. As características estruturais representam os interrelacionamentos entre os componentes do sistema. A parte
dinâmica representa as funcionalidades do sistema e como elas são realizadas em tempo de execução. Ambas características devem ser modeladas em alto 
(modelos representando o domínio do problema) e baixo nível de abstração (modelos próximos ao nível de código), obtendo assim uma visão do todo e da
parte \cite{silva:00}.

Na fase de análise, primeiramente deve-se descobrir qual o problema do cliente e eliciar quais são os requisitos do sistema. Cada requisito
pode ser classificado como um interesse do cliente a fim de atingir um objetivo final \cite{Laddad:2003:AAP:993468}. Ao final do desenvolvimento,
após a modelagem nas fases de análise e projeto e o código nas fases de implementação e testes, todos interesses do cliente devem estar resolvidos, tendo
como resultado um sistema completo e funcional. Os interesses podem ser separados em dois tipos \cite{Laddad:2003:AAP:993468}:

\begin{itemize}
  \item Interesses núcleo: Capturam uma funcionalidade principal e impactam apenas uma parte do sistema.
  \item Interesses entrecortantes (\textit{Crosscutting concerns}): Capturam uma funcionalidade que impacta uma ou mais partes do sistema. 
\end{itemize}

A POO pode ser utilizada para implementação de ambos tipos de interesses. O problema é que para sistemas complexos, há uma tendência de misturar
interesses entrecortantes com interesses núcleo, prejudicando a manutenabilidade e compreensão do código. Conclui-se que a POO tem limitações para
implementar extensões para comportamentos que impactam várias partes de um sistema: os interesses entrecortantes
\cite{Kiczales97aspect-orientedprogramming}. A Programação Orientada a Aspectos (POA) \sigla{POA}{Programação Orientada a Aspectos} é uma extensão a
POO que permite uma implementação mais elegante para interesses entrecortantes.

O objetivo da POA é a modularização dos interesses entrecortantes para que os mesmos fiquem separados dos módulos que implementam os interesses
núcleo da aplicação \cite{Laddad:2003:AAP:993468}. Os interesses devem ser separados em módulos em todas as fases do desenvolvimento: análise,
projeto, implementação e testes. Assim, obtém-se um mapeamento direto de um interesse, facilitando a manutenção e compreensão de um sistema
desenvolvido utilizando aspectos. A separação de interesses é uma excelente prática para a construção de sistemas complexos, pois quanto maior o
número de interesses, maior a complexidade para implementá-los. Com a separação de interesses, cada módulo representa um interesse, que pode ser
implementado separadamente, diminuindo a complexidade de implementação \cite{Jacobson:2004:ASD:1062430}.

\section{Definição do Problema}

Segundo \cite{silva:00}, na modelagem de programas orientados a objetos podem-se identificar quatro pontos de vista essenciais: estrutural e
comportamental de sistema e estrutural e comportamental de classe. Um sistema modelado pelos quatro pontos de vista permite a compreensão e manutenção
do sistema e contribui para a geração de código em qualquer linguagem de programação, sendo considerada assim uma modelagem completa \cite{silva:07}.

Algumas abordagens já foram propostas para modelagem de sistemas orientados a aspectos com UML (Unified Modeling Language) \cite{uml:05}. Um
grupo de propostas estende o meta-modelo da linguagem \cite{Kienzle:2009:AMM:1509239.1509252} \cite{theme:04} \cite{Klein:2007:WMA:1805812.1805819} \cite{Jacobson:2004:ASD:1062430} \cite{france:06},
introduzindo novas construções para representar a estrutura e o comportamento de aplicações orientadas a aspectos. Outro conjunto de propostas estende
a UML através de um perfil UML \cite{Evermann:2007:MSP:1229375.1229379} \cite{Cottenier06themotorola} \cite{Cui:2009:MIA:1529282.1529377}, o qual
pode ser utilizado em ferramentas CASE que suportem a importação de perfis. No entanto, não existe nenhuma abordagem para modelagem de aspectos que
represente a estrutura e a dinâmica de um sistema e que forneça subsídios para visualização do efeito dos aspectos no sistema, realizando a
composição de modelos de aspectos com modelos núcleo automaticamente. Além disso, é importante que a modelagem permita representar as características
inerentes à POA, como a possibilidade de capturar múltiplos pontos de execução de um programa com apenas uma expressão regular. A maior parte das propostas se
limitam a modelar apenas parte destas características.

\section{Objetivos}

\subsection{Objetivo Geral}

Propor uma abordagem para modelar sistemas orientados a aspectos nas fases de análise e projeto com a segunda versão da UML. Esta modelagem deve
representar a estrutura e a dinâmica de um sistema, permitir automatizar parte do processo de modelagem, possibilitando a alternância de visões da
dinâmica do sistema (com e sem explicitação dos aspectos) e, com isso, facilitar a compreensão e manutenção dos modelos.

\subsection{Objetivos Específicos}

\begin{itemize}
  \item Modelar as características inerentes a aspectos com a segunda versão da UML.
  \item Representar a estrutura e a dinâmica de sistemas orientados a aspectos, fornecendo subsídios para a composição de modelos e facilitando 
  a compreensão e manutenção do sistema.
  \item Permitir a alternância de visões do comportamento dinâmico de um sistema orientado a aspectos, visualizando o efeito de modelos de
  interesses entrecortantes (aspectos) em modelos núcleo.
\end{itemize}

\section{Hipótese de Pesquisa}

É possível modelar todas as características de programas orientados a aspectos com a segunda versão de UML, bem como é possível manipular esta
modelagem em um procedimento algorítmico capaz de explicitar e ocultar dinamicamente os aspectos da modelagem.

\section{Justificativa}

Tratando-se de POO, uma modelagem pode ser considerada completa se modelar os quatro pontos de vista fundamentais \cite{silva:00}. Em relação a
sistemas orientados a aspectos, além da modelagem estrutural e comportamental, definem-se dois requisitos importantes para se obter uma modelagem
completa:

\begin{enumerate}
  \item Representação das características inerentes à programas orientados a aspectos como: \textit{wildcards}, pontos de corte, avisos e
  introduções. 
  \item Representação da estrutura e da dinâmica do sistema para permitir a composição em nível de modelo de interesses núcleo com interesses
  entrecortantes (aspectos).
\end{enumerate}

Sabendo que a modelagem estrutural e dinâmica de um sistema facilita a compreensão e manutenção e também contribui para a composição de modelos,
conclui-se que este trabalho justifica-se pelo uso de artefatos que representem a estrutura e a dinâmica de um sistema, automatizando parte do processo de modelagem. 
A compreensão do fluxo de execução de programas orientados a aspectos é difícil, assim, justifica-se a implementação de um visualizador do efeito de
aspectos em modelos núcleo. Sabe-se também que a segunda versão da UML é um padrão e é amplamente utilizada na modelagem de aplicações, sendo assim,
justifica-se que os modelos propostos estejam de acordo com a especificação proposta por esta linguagem, estendendo a linguagem através de um perfil UML. Em 
relação a aspectos, a modelagem de \textit{wildcards} é fundamental para possibilitar a captura dos pontos do sistema que serão estendidos de maneira
praticável. Logo, justifica-se a preocupação em modelar todas as características inerentes à POA.

\section{Método de Pesquisa}

A realização deste trabalho é composta pelos seguintes passos:

\begin{enumerate}
  \item Análise de trabalhos correlatos;
  \item Proposta para a especificação de sistemas orientados a aspectos utilizando a segunda versão da UML;
  \item Implementar o suporte para modelagem proposta no ambiente SEA \cite{silva:00}, gerando um produto deste trabalho, que é o suporte a
  modelagem de programas orientados a aspectos no ambiente SEA, permitindo a alternância de visões dos interesses modelados automaticamente;
  \item Realizar um estudo de caso com a proposta de modelagem, realizando a composição de aspectos em nível de modelo e utilizando o visualizador de
  interesses.
  \item Comparar a abordagem proposta por esta dissertação com os trabalhos correlatos, explicitando os diferentes requisitos cumpridos por cada
  proposta.
\end{enumerate}

\section{Classificação da Pesquisa}

A classificação desta pesquisa foi realizada baseando-se nas informações sobre metodologia de pesquisa de \cite{silva-menezes:01}. Em relação a
natureza, classifica-se esta pesquisa como \textbf{aplicada}, pois ela tem como objetivo gerar conhecimentos para aplicação prática em cima de
problemas específicos na área de modelagem de sistemas orientados a aspectos. A abordagem do problema utilizada nesta dissertação é
\textbf{qualitativa}, com o pesquisador realizando a análise dos dados indutivamente, onde o processo e seu significado são os
focos principais da abordagem. Em relação aos objetivos, classifica-se esta pesquisa como \textbf{exploratória}, realizando a análise dos trabalhos
correlatos e identificando quais parâmetros são implementados por cada proposta para modelagem de sistemas orientados a aspectos. Finalmente, em
relação aos procedimentos técnicos utilizados, classifica-se esta pesquisa como \textbf{pesquisa bibliográfica}, pois foi elaborada a partir de
material já publicado sobre a modelagem de sistemas orientados a aspectos. Esta pesquisa também tem um caráter \textbf{experimental} com a
implementação do protótipo para modelagem de aspectos no ambiente SEA.

\section{Organização desta dissertação}

Esta dissertação está organizada da seguinte maneira: o presente capítulo descreveu a motivação para a modelagem de sistemas orientados a aspectos,
objetivos, hipótese, método de pesquisa e justificativa. O capítulo {2} apresenta a fundamentação teórica com conceitos de programação orientada a
aspectos e da linguagem AspectJ, conceitos de análise e projetos de sistemas e modelagem e meta-modelagem com a segunda versão da UML. O capítulo {3} 
discute sobre os trabalhos correlatos, realizando uma comparação entre abordagens. O capítulo {4} descreve a metodologia de modelagem
para a especificação de sistemas orientados a aspectos, incluindo o algoritmo para composição de aspectos. O capítulo {5} mostra dois estudos de caso
que verificam a aplicabilidade da abrodagem proposta. O capítulo {6} apresenta uma comparação entre diferentes abordagens para modelagem de sistemas orientados a aspectos. 
Finalizando, o capítulo {7} discute sobre as conclusões deste trabalho.
