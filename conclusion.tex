\chapter{Conclusão}

Este trabalho propõe uma abordagem para modelagem de sistemas orientados a aspectos usando UML, através de um perfil UML que permite especificar a
estrutura e o comportamento de sistemas orientados a aspectos. Um produto deste trabalho é o visualizador de aspectos no
ambiente SEA: SEA/Aspect. Esta ferramenta realiza a composição entre modelos núcleo e entrecortantes automaticamente, permitindo a
visualização da dinãmica de aspectos em um sistema.

\section{Principais Resultados}

Uma das hipóteses de pesquisa salienta que é possível modelar todas as características de sistemas orientados a aspectos com UML. Os seguintes
resultados conprovam esta hipótese:

\begin{itemize}
	\item Definição de um perfil dentro dos padrões da UML, o qual pode ser adicionado em ferramentas CASE que suportem a importação
	de perfis; 
	\item Representação de características da POA, como a captura de múltiplos pontos de junção e a representação do comportamento de
	aspectos com os diagramas de máquina de estado para representar os pontos de corte e os diagramas de sequência para representar os avisos. Este é o primeiro
	trabalho que utiliza invariantes de estado para conectar a satisfação de pontos de corte com o disparo de avisos. A modelagem proposta também permite
	realizar a composição de avisos do tipo \textit{around} e representar pontos de corte que capturam o fluxo de execução de um sistema (\textit{cflow}
	e \textit{cflowbelow};
\end{itemize}

Em relação a automatização e manipulação da modelagem em um procedimento algorítmico para possibilitar a visualização da dinâmica de aspectos, os
resultados a seguir comprovam a hipótese:

\begin{itemize}
\item Possibilidade de composição automática de pontos de corte, utilizando o diagrama de máquina de estados;
\item Possibilidade de composição automática e visualização do efeito dos modelos entrecortantes (aspectos) nos modelos núcleo, que facilita
a compreensão e manutenção de sistemas orientados a aspectos. A ferramenta que permite a composição e a alternância de visões pode ser adicionada a ferramentas CASE 
que suportem extensão funcional, que é o caso do ambiente SEA.
\end{itemize}

\section{Produtos de Trabalho}

Os produtos resultantes deste trabalho de pesquisa são:

\begin{itemize}
  \item Um perfil UML que representa as características de sistemas orientados a aspectos. Este perfil pode ser importado em ferramentas CASE
  que suportem a importação de perfis.
  \item Uma ferramenta que permite modelar um sistema orientado a aspectos, representando as características inerentes à POA,
  permitindo realizar a composição de aspectos e visualizar o efeito dos mesmos em modelos núcleo;
  \item Artigo submetido e aceito: International Conference on Computers and Their Applications (CATA) \cite{ghilardi_cata:13};
  \item Artigo submetido e aceito: Congreso Iberoamericano en Ingeniería de Software (CibSE) \cite{ghilardi_cibse:13};
\end{itemize}

\section{Limitações}

O presente trabalho tem as seguintes limitações:

\begin{itemize}
  \item O perfil UML para especificação de aspectos pode ser utilizado por outras ferramentas CASE. No entanto, a ferramenta para composição de
  modelos só pode ser utilizada em ferramentas que suportem a extensão funcional, através de um plug-in, por exemplo.
  \item A aplicabilidade da abordagem proposta é verificada em um exemplo proposto por Jacobson \cite{Jacobson:2004:ASD:1062430}. Nenhum
  caso real foi modelado com o perfil proposto.
  \item Embora um perfil UML pode ser importado em diferentes ferramentas CASE comerciais, este trabalho não conseguiu realizar a importação do perfil
  (.xmi) na ferramenta de modelagem Visual Paradigm \cite{VisualParadigm11}. Algumas ferramentas comerciais não seguem o padrão para importação de
  perfis, o que demanda uma investigação para suportar todas as ferramentas CASE.	
\end{itemize}

\section{Trabalhos Futuros}

Como trabalho futuro pode-se investigar a possibilidade de geração de código em AspectJ, a partir da especificação e composição dos modelos núcleo e
entrecortantes. Pode-se também realizar a composição entre modelos compostos de interesses núcleos com interesses entrecortantes, realizando uma
operação de \textit{merge} entre os modelos (tanto comportamentais quanto estruturais). Além disso, a aplicação da proposta de modelagem em um sistema
real pode fornecer mais resultados sobre a aplicabilidade da abordagem em casos mais complexos.

O suporte a importação do perfil UML para modelagem de sistemas orientados a aspectos em ferramentas CASE é um trabalho futuro importante,
pois remove a dependência de utilização do perfil com a ferramenta de modelagem SEA/Aspect.
