\chapter{Conclusão}

Este trabalho propõe uma abordagem para modelagem de sistemas orientados a aspectos usando UML, através de um perfil UML que permite especificar a
estrutura e o comportamento de sistemas orientados a aspectos. Um importante produto deste trabalho é o visualizador de aspectos no
ambiente SEA: SEA/Aspect. A ferramenta SEA/Aspect realiza a composição entre modelos núcleo e entrecortantes automaticamente, permitindo a
visualização da dinãmica de aspectos em um sistema.

\section{Principais Resultados}

As principais vantagens da abordagem proposta em relação as outras são: 

\begin{itemize}
\item Representação das características importantes da POA, como a captura de múltiplos pontos de junção e a representação da dinâmica de aspectos
com os diagramas de máquina de estado para representar pontos de corte e os diagramas de sequência para representar avisos. Este é o primeiro
trabalho que utiliza invariantes de estado para conectar a satisfação de pontos de corte com o disparo de avisos;
\item Possibilidade de composição automática de pontos de corte utilizando o diagrama de máquina de estados;
\item Definição de um perfil dentro dos padrões da UML, o qual pode ser adicionado em ferramentas CASE que suportem a importação
de perfis; 
\item Possibilidade de composição automática e visualização dinâmica do efeito dos modelos entrecortantes (aspectos) nos modelos núcleo, que facilita
a compreensão e manutenção de sistemas orientados a aspectos. A ferramenta que permite a composição e a alternância de visões pode ser adicionada a ferramentas CASE 
que suportem extensão funcional, que é o caso do ambiente SEA.
\end{itemize}

\section{Produtos de Trabalho}

Os produtos resultantes deste trabalho de pesquisa são:

\begin{itemize}
  \item Um perfil UML que representa as características de sistemas orientados a aspectos. Este perfil pode ser importado em ferramentas CASE
  que suportem a importação de perfis.
  \item Uma ferramenta (SEA/Aspect) que permite modelar um sistema orientado a aspectos, representando as características inerentes à POA,
  permitindo realizar a composição de aspectos e visualizar o efeito dos mesmos em modelos núcleo;
  \item Artigo submetido e aceito: International Conference on Computers and Their Applications (CATA) \cite{ghilardi_cata:13};
  \item Artigo submetido e aceito: Congreso Iberoamericano en Ingeniería de Software (CibSE) \cite{ghilardi_cibse:13};
\end{itemize}

\section{Limitações}

O presente trabalho tem as seguintes limitações:

\begin{itemize}
  \item A especificação de aspectos é baseada nas construções disponíveis na linguagem AspectJ. Logo, a modelagem proposta por este trabalho é
  dependente de plataforma.
  \item O perfil UML para especificação de aspectos pode ser utilizado por outras ferramentas CASE. No entanto, a ferramenta para composição de
  modelos só pode ser utilizada em ferramentas que suportem a extensão funcional, através de um plug-in, por exemplo.
  \item A aplicabilidade da abordagem proposta é verificada em dois estudos de caso propostos por Jacobson \cite{Jacobson:2004:ASD:1062430}. Nenhum
  caso real foi modelado com o perfil proposto.
\end{itemize}

\section{Trabalhos Futuros}

Como trabalho futuro sugere-se realizar um novo estudo de caso com um caso real de alguma aplicação, contendo pontos de corte mais complexos, como os
que tratam do fluxo de execução (\textit{cflow}, \textit{cflowbelow}). Este estudo de caso pode fornecer resultados mais concretos, com
o objetivo de verificar que o visualizador de aspectos facilita a compreensão e manutenção de sistemas orientados a aspectos. Também pode-se
investigar a possibilidade de geração de código em AspectJ, a partir dos modelos núcleo e entrecortantes.
