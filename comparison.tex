\chapter{Compara��o de Abordagens}

There are several proposals for modeling aspect-oriented applications in the AOM
research area. It is important to identify some variation points between these
approaches. This paper makes a comparison between the related approaches by comparing 
some points like: 

\begin{table*}
	\centering
	\begin{tabular}{ | c | c | c | c | c | c | c |}
		\hline
		  & Klein \cite{Klein:2007:WMA:1805812.1805819} & 
		  Evermann \cite{Evermann:2007:MSP:1229375.1229379} &
		  Kienzle \cite{Kienzle:2009:AMM:1509239.1509252} &
		  Jacobson \cite{Jacobson:2004:ASD:1062430} &
		  Cui \cite{Cui:2009:MIA:1529282.1529377} &
		  Zhang \cite{Zhang:2012:WSA:2162049.2162080} 
		  \\
		\hline
		 Extension Type & Meta-model & UML Profile & Meta-model & Meta-model &
		 UML Profile & Meta-model \\
		\hline
		 Represents Structure? & No & Yes & Yes & Yes & No & No\\
		\hline
		 Represents Behavior? & Yes & No & Yes & Yes & Yes & Yes\\
		\hline
		 Automatic Composition?  & Yes & Yes & Yes & No & Yes & Yes\\
		\hline
		 AOP Representation & Partial & Partial & Total & Total & Partial & Partial\\
		\hline
		 Visualization Toggling? & No & No & No & No & No & No\\
		\hline
		 Level of Abstraction & High & High & High & High & High & High \\
		\hline
	\end{tabular}
	\caption{Table Comparison between AOM approaches (1).}
	\label{tab:comparison_table_1}
\end{table*} 

\begin{table*}
	\centering
	\begin{tabular}{ | c | c | c | c | c | c | c |}
		\hline
		  & Cottenier \cite{Cottenier06themotorola} &
		  Clarke \cite{theme:04} &
		  France \cite{france:06} &
		  Whittle \cite{Whittle:2008} &
		  Ghilardi
		  \\
		\hline
		 Extension Type & UML Profile & Meta-model & Meta-model & UML
		 Profile & UML Profile\\
		\hline
		 Represents Structure? & Yes & Yes & Yes & Yes & Yes\\
		\hline
		 Represents Behavior? & Yes & Yes & No & Yes & Yes\\
		\hline
		 Automatic Composition? & Yes & Yes & Yes & Yes & Yes\\
		\hline
		 AOP Representation & Total & Total & Partial & Total & Total\\
		\hline
		 Visualization Toggling? & Yes & No & No & No & Yes\\
		\hline
		 Level of Abstraction & Low & High & High & High & High\\
		\hline
	\end{tabular}
	\caption{Table Comparison between AOM approaches (2).}
	\label{tab:comparison_table_2}
\end{table*} 

\begin{itemize}
  \item \textbf{Extension Type:} Some approaches create a lightweight extension
  to UML using a profile that can be reused in any CASE tool. Others changes the UML
  meta-model, diminishing the possibility of reuse.
  \item \textbf{Represents Structure?} The structural of aspect-oriented
  applications is represented? Modeling introductions of fields or changes in the inheritance
  hierarchy.
  \item \textbf{Represents Behavior?} The behavior of aspect-oriented
  applications is represented? Modeling pointcuts and advices, representing the execution points that will be
  captured and the behavior that will be executed in these points.
  \item \textbf{Automatic Composition?} Allows the automatic composition of
  models, without effort from the developer.
  \item \textbf{AOP Representation:} Represents the inherent characteristics of
  the aspect-oriented approach like the modularization of concerns and the
  possibility to capture multiple join points in the same definition
  (wildcards).
  \item \textbf{Visualization Toggling:} Allows the toggling of views,
  visualizing the core concern, the crosscutting concern or both intertwined.
  \item \textbf{Level of Abstraction:} Represents the application in low or
  high-level of abstraction.
\end{itemize}

The comparison table can be view in the tables \ref{tab:comparison_table_1} and
\ref{tab:comparison_table_2}. It is important to highlight the approaches that
use a UML profile to represent aspect-orientead applications. An extension
through a profile can be reused in any CASE tool, which is an important
advantage of the approaches that extends the language with this mechanism. The
automatic composition of models is a desired functionality, because it decreases
the time spent in the modeling by automating part of it. It is
important to represent all the characteristics of AOP using structural
and behavioral diagrams. Finally, approaches that allows the dynamic
visualization of concerns give an additional advantage, because they facilitates
the understanding of aspect-oriented applications. High levels of abstraction are desired,
because it facilitates the understanding of the system being modeled. A modeling
in low-level of abstraction has constructions too close to the code, which
eases code generation, but difficults the system understanding.